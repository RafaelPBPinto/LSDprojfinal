\documentclass{report}
\usepackage[T1]{fontenc} % Fontes T1
\usepackage[utf8]{inputenc} % Input UTF8
\usepackage[backend=biber, style=ieee]{biblatex} % para usar bibliografia
\usepackage{csquotes}
\usepackage[portuguese]{babel} %Usar língua portuguesa
\usepackage{blindtext} % Gerar texto automaticamente
\usepackage[printonlyused]{acronym}
\usepackage{hyperref} % para autoref
\usepackage{graphicx}


\bibliography{bibliografia}


\begin{document}
%%
% Definições
%
\def\titulo{Projeto Final de LSD }
\def\data{17 de junho de 2021}
\def\autores{Guilherme Craveiro, Rafael Pinto}
\def\autorescontactos{(103574) gjscraveiro@ua.pt, (103379) rafaelpbpinto@ua.pt}
\def\versao{Versão 1}
\def\departamento{Departamento de Eletrónica, Telecomunicações e Informática}
\def\empresa{Universidade de Aveiro}
\def\logotipo{ua.pdf}
%
%%%%%% CAPA %%%%%%
%
\renewcommand{\contentsname}{Índice}
\begin{titlepage}

\begin{center}
%
\vspace*{50mm}
%
{\Huge \titulo}\\ 
%
\vspace{10mm}
%
{\Large \empresa}\\
%
\vspace{10mm}
%
{\LARGE \autores}\\ 
%
\vspace{30mm}
%
\begin{figure}[h]
\center
\includegraphics{\logotipo}
\end{figure}
%
\vspace{30mm}
\end{center}
%
\begin{flushright}
\versao
\end{flushright}
\end{titlepage}

%%  Página de Título %%
\title{%
{\Huge\textbf{\titulo}}\\
{\Large \departamento\\ \empresa}
}
%
\author{%
    \autores \\
    \autorescontactos
}
%
\date{\data}
%
\maketitle

\pagenumbering{roman}

%%%%%% RESUMO %%%%%%
\begin{abstract}
Este relatório tem como objetivo explicar o funcionamento da máquina automática de oferta de produtos desenvolvida no nosso projeto. Para que a máquina funcionasse foi necessário a implementação de código em \ac{vhdl} e procedeu-se vários testes e à análise dos mesmos. 
\end{abstract}

\tableofcontents
% \listoftables     % descomentar se necessário
% \listoffigures    % descomentar se necessário


%%%%%%%%%%%%%%%%%%%%%%%%%%%%%%%
\clearpage
\pagenumbering{arabic}

%%%%%%%%%%%%%%%%%%%%%%%%%%%%%%%%
\chapter{Introdução}
\label{chap.introducao}

\ac{vhdl} é uma linguagem usada para modelar o comportamento e a estrutura de sistemas digitais em, por exemplo, \ac{fpga}. De forma muito resumida, \ac{fpga} é uma matriz de blocos lógicos interligados de modo inteligente que podem ser reprogramados para a aplicação desejada.

\chapter{Metodologia}
\label{chap.metodologia}
Descreve os métodos utilizados para obtenção de resultados.

Neste esqueleto de relatório aproveitamos este capítulo para exemplificar
como se usam alguns elementos de {\LaTeX}.

\section{Exemplos}

\subsection{Utilização de acrónimos}
\label{sec.util}
Esta é a primeira invocação do acrónimo \ac{ua}.
E esta é a segunda: \ac{ua}.


Outras duas referências a \ac{miect}
e \ac{miect}.

\subsection{Referências bibliográficas}
Informação relativa à estrutura formal de um relatório pode ser obtida
na página do \ac{glisc}\cite{glisc}.

Como foi apresentado na \autoref{sec.util}...

\chapter{Resultados}
\label{chap.resultados}
Descreve os resultados obtidos.

\chapter{Análise}
\label{chap.analise}
Analisa os resultados.

\chapter{Conclusões}
\label{chap.conclusao}
Apresenta conclusões.

\chapter*{Contribuições dos autores}
Resumir aqui o que cada autor fez no trabalho.
Usar abreviaturas para identificar os autores,
por exemplo AS para António Silva.
No fim indicar a percentagem de contribuição de cada autor.

%%%%%%%%%%%%%%%%%%%%%%%%%%%%%%%%%
\chapter*{Acrónimos}
\begin{acronym}
\acro{ua}[UA]{Universidade de Aveiro}
\acro{miect}[MIECT]{Mestrado Integrado em Engenharia de Computadores e Telemática}
\acro{lei}[LEI]{Licenciatura em Engenharia Informática}
\acro{glisc}[GLISC]{Grey Literature International Steering Committee}
\acro{lsd}[LSD]{Laboratórios de Sistemas Digitais}
\acro{vhdl}[VHDL]{VHSIC Hardware Description Language}
\acro{fpga}[FPGA]{Field Programmable Gate Array}
\end{acronym}


%%%%%%%%%%%%%%%%%%%%%%%%%%%%%%%%%
\printbibliography

\end{document}